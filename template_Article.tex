\documentclass[]{article}
\usepackage{titling}
\usepackage{xepersian}
\settextfont{Arial}

%opening 

\title{ الهی به توکل نام اعظمت  }

\author{
فاز اول پروژه اصول طراحی پایگاه داده
 }

\begin{document}
\maketitle

\begin{center}
\textbf{سیستم خرید آنلاین لباس}
\end{center}


\begin{center}
\textbf{میثم عسلی 401521435 - ابوالفضل شهیدی 401521381}
\end{center}

\newpage
\tableofcontents
\newpage
\section{شرح پروژه}
مدیر فروشگاه با ثبت نام در سایت و ثبت شعب مختلف آن, می تواند اطلاعاتی نظیر آدرس, شماره تلفن, ساعت کار , محصولات  را به نمایش کاربران بگذارد.هر محصول شامل ویژگی هایی نظیر سایز, رنگ, قیمت, درصد رضایت و تعداد باقی مانده در انبار می باشد. کاربران با ثبت نام در سایت و افزایش  اعتبار خود, می توانند سفارش خود را ثبت کنند.
هر سفارش می تواند یک یا چندین آیتم باشد و هر آیتم شامل یک محصول و تعداد است.
همچنین کاربران می تواننند نظر خود را راجب محصولی که قبلا خریداری کرده اند ثبت کنند. 
\section{موجودیت ها}






\subsection{فروشگاه}
این موجودیت یکی از مهم ترین موجودیت های سیستم است که شامل یک یا چند شعبه است.
\begin{itemize}
\item کد فروشگاه
\item نام فروشگاه
\item تاریخ ثبت  
\item شناسه مدیر

\end{itemize}








\subsection{شعبه}
هر شعبه متعلق به یک فروشگاه می باشد و شامل محصولاتی می باشد.
\begin{itemize}
\item نام شعبه
\item کد شعبه
\item کد فروشگاه
\item تاریخ ثبت  
\item کد پستی
\item کشور
\item شهر
\item آدرس
\item شماره تلفن
\end{itemize}






\subsection{مشتری}
\begin{itemize}
\item شماره مشتری
\item نام کاربری
\item نام
\item نام خانوادگی
\item کد ملی
\item جنسیت
\item آدرس
\item کد پستی
\item شماره موبایل
\item ایمیل
\item رمز عبور
\end{itemize}







\subsection{آیتم}
هر آیتم متعلق به یک سبد خرید است

\begin{itemize}
\item شناسه آیتم
\item شناسه محصول
\item تعداد
\item قیمت اولیه \newline
حاصل ضرب قیمت محصول و تعداد آن
\item تخفیف \newline
حاصل ضرب تخفیف محصول در قیمت اولیه
\item قیمت نهایی \newline
قیمت اولیه منهای تخفیف
\end{itemize}








\subsection{سبد خرید}
هر سبد خرید متعلق به یک سفارش و شامل چندین آیتم است و اگر نهایی شود یک سفارش جدید ایجاد می شود.
\begin{itemize}
\item شناسه سبد خرید
\item شناسه سفارش
\item نهایی شده
\begin{enumerate}
\item بله \newline
پس از پرداخت هزینه نهایی می شود و یک سفارش جدید ایجاد می شود.
\item خیر \newline
\end{enumerate}

\end{itemize}






\subsection{سفارش}
هر سفارش شامل یک سبد خرید است.
\begin{itemize}
\item شماره سفارش
\item شماره مشتری
\item شناسه سبد خرید
\item تاریخ ثبت سفارش
\item زمان تحویل
\item تخفیف کل
حاصل جمع تخفیف تمامی آیتم ها
\item قیمت اولیه محصولات
حاصل جمع قیمت اولیه همه آیتم ها
\item قیمت نهایی محصولات
قیمت اولیه منهای تخفیف کل
\item هزینه ارسال 
\item مبلغ نهایی \newline
حاصل جمع هزینه ارسال و قیمت نهایی محصولات
\end{itemize}









\subsection{محصول}
\begin{itemize}
\item شماره محصول
\item عکس
\item سایز
\item رنگ
\item قیمت
\item درصد رضایت
\item باقی مانده
\item درصد تخفیف \newline
عددی بین 0 و 100
\item  نوع لباس
\begin{enumerate}
\item مردانه
\item زنانه
\item بچگانه
\end{enumerate}
\item کشور تولید کننده
\item جنس
\begin{enumerate}
\item پنبه
\item الیاف طبیعی
\item نخ
\item پلی استر
\end{enumerate}
\item توضیح محصول
\end{itemize}







\subsection{نظرات}
هر کابری که قبلا محصول را خریداری کرده است می تواند نظر خود را درباره آن محصول ثبت نماید.
\begin{itemize}
\item شماره نظر
\item شماره مشتری
\item عنوان نظر
\item نقاط قوت  
\item نقاط ضعف
\item متن نظر
\item دیدگاه اجمالی
\begin{enumerate}
\item خیلی بد
\item بد
\item معمولی
\item  خوب
\item  خیلی خوب
\end{enumerate}
\item ارسال به صورت ناشناس
\begin{enumerate}
\item بله
\item خیر
\end{enumerate}
\end{itemize}





\subsection{مدیر}
مدیر نماینده یک فروشگاه محسوب می شود و وظیفه ثبت فروشگاه و شعب آن را دارد.
\begin{itemize}
\item شناسه مدیر
\item نام کابری
\item نام
\item نام خانوادگی
\item کد ملی
\item جنسیت
\item ایمیل
\item تاریخ ثبت نام
\item رمز عبور
\end{itemize}







\section{گزارش ها}



\end{document}
}



