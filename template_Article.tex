\documentclass[]{article}
\usepackage{titling}
\usepackage{xepersian}
\settextfont{Arial}

%opening 

\title{ الهی به توکل نام اعظمت  }

\author{
فاز اول پروژه اصول طراحی پایگاه داده
 }

\begin{document}
\maketitle

\begin{center}
\textbf{سیستم خرید آنلاین لباس}
\end{center}

\begin{center}
\textbf{ استاد درس : دکتر خنجری   }
\end{center}
\begin{center}
\textbf{ اعضا : میثم عسلی 401521435 - ابوالفضل شهیدی 401521381}
\end{center}

\newpage
\tableofcontents
\newpage
\section{شرح پروژه}
مدیر فروشگاه ابتدا باید در سیستم ثبت نام کند. پس از آن می تواند فروشگاه خود را ثبت کند. در مرحله بعدی می تواند شعبه های فروشگاه را ثبت کند. هر شعبه شامل محصولاتی می باشد که تعداد باقی مانده, میزان تخفیف, نظرات آن قابل مشاهده است. 
مشتری باید ابتدا در سیستم ثبت نام کند. بعد از آن می تواند سفارشات خود را ثبت کند.
هر سفارش شامل متعلق به یک مشتری و شامل یک سبد خرید است. 
هر سبد خرید متعلق به یک سفارش و شامل چندین آیتم است.
همچنین هر آیتم شامل یک محصول و تعداد آن است.
افرادی که محصولی را خریداری کرده اند می توانند نظر خود درباره آن محصول را ثبت نمایند. 


\section{موجودیت ها}



\subsection{فروشگاه}
این موجودیت یکی از مهم ترین موجودیت های سیستم است که شامل یک یا چند شعبه است و مدیر آن با ثبت نام در سیستم, فروشگاه را ثبت می کند.
\begin{itemize}
\item کد فروشگاه
\item نام فروشگاه
\item تاریخ ثبت  
\item شناسه مدیر
\item شعبه ها
\end{itemize}






\subsection{شعبه}
هر شعبه متعلق به یک فروشگاه می باشد و شامل محصولاتی می باشد.
\begin{itemize}
\item نام شعبه
\item کد شعبه
\item کد فروشگاه
\item تاریخ ثبت  
\item کد پستی
\item کشور
\item شهر
\item آدرس
\item شماره تلفن
\item محصولات
\end{itemize}



\subsection{محصول}
هر محصولی با شناسه از بقیه متمایز می شود و متعلق به یک شعبه فروشگاه می باشد. هر محصول قسمت نظرات دارد.
\begin{itemize}
\item شناسه محصول
\item شناسه شعبه
\item عکس
\item سایز
\item رنگ
\item قیمت
\item درصد رضایت
\item باقی مانده
\item درصد تخفیف \newline
عددی بین 0 و 100
\item  نوع لباس
\begin{enumerate}
\item مردانه
\item زنانه
\item بچگانه
\end{enumerate}
\item کشور تولید کننده
\item جنس
\begin{enumerate}
\item پنبه
\item الیاف طبیعی
\item نخ
\item پلی استر
\end{enumerate}

\item توضیح محصول

\end{itemize}



\subsection{مشتری}
هر مشتری با ثبت نام در سیستم می تواند سفارشات خود را ثبت کند. هر مشتری شامل چندین سفارش است. 
\begin{itemize}
\item شماره مشتری
\item نام کاربری
\item نام
\item نام خانوادگی
\item کد ملی
\item جنسیت
\item آدرس
\item کد پستی
\item شماره موبایل
\item ایمیل
\item رمز عبور
\item سفارش ها
\item نظرات
\item تراکنش ها
\end{itemize}





\subsection{آیتم}
هر آیتم یک محصول دارد و متعلق به یک سبد خرید می باشد که با شناسه از بقیه متمایز می شود.

\begin{itemize}
\item شناسه آیتم
\item شناسه محصول
\item تعداد
\item قیمت اولیه \newline
حاصل ضرب قیمت محصول و تعداد آن
\item تخفیف \newline
حاصل ضرب تخفیف محصول در قیمت اولیه
\item قیمت نهایی \newline
قیمت اولیه منهای تخفیف
\end{itemize}








\subsection{سبد خرید}
هر سبد خرید شامل چندین آیتم است و اگر نهایی شود یک سفارش جدید ایجاد می شود.
\begin{itemize}
\item شناسه سبد خرید
\item شناسه سفارش
\item شناسه مشتری
\item نهایی شده
\begin{enumerate}
\item بله \newline
پس از پرداخت هزینه نهایی می شود و یک سفارش جدید ایجاد می شود.
\item خیر \newline
\end{enumerate}
\item تخفیف کل
حاصل جمع تخفیف همه آیتم ها
\item قیمت اولیه محصولات
حاصل جمع قیمت اولیه همه آیتم ها
\item آیتم ها
\end{itemize}






\subsection{سفارش}
هر سفارش متعلق به یک مشتری و شامل یک سبد خرید است.
\begin{itemize}
\item شناسه سفارش
\item شناسه مشتری
\item شناسه سبد خرید
\item تاریخ ثبت سفارش
\item  تاریخ تحویل مرسوله

\item قیمت نهایی محصولات
قیمت اولیه سبد خرید منهای تخفیف کل سبد خرید 
\item هزینه ارسال 
\item مبلغ نهایی \newline
حاصل جمع هزینه ارسال و قیمت نهایی محصولات
\end{itemize}
















\subsection{نظرات}
هر کابری که قبلا محصول را خریداری کرده است می تواند نظر خود را درباره آن محصول ثبت نماید.
\begin{itemize}
\item شناسه نظر
\item شناسه مشتری
\item عنوان نظر
\item نقاط قوت  
\item نقاط ضعف
\item متن نظر
\item دیدگاه اجمالی
\begin{enumerate}
\item خیلی بد
\item بد
\item معمولی
\item  خوب
\item  خیلی خوب
\end{enumerate}
\item ارسال به صورت ناشناس
\begin{enumerate}
\item بله
\item خیر
\end{enumerate}
\end{itemize}





\subsection{مدیر}
مدیر نماینده یک فروشگاه محسوب می شود و وظیفه ثبت فروشگاه و شعب آن را دارد.
\begin{itemize}
\item شناسه مدیر
\item شناسه فروشگاه
\item نام کابری
\item نام
\item نام خانوادگی
\item کد ملی
\item جنسیت
\item ایمیل
\item تاریخ ثبت نام
\item رمز عبور 
\end{itemize}







\section{گزارش ها}
\begin{enumerate}
\item فیلتر کردن محصولات بر اساس قیمت
\item بدست آوردن محصولات یک فروشگاه
\item بدست آوردن نظرات ثبت شده برای یک محصول 
\item بدست آوردن تمام سفارش های یک شعبه در یک بازه خاص
\item بدست آوردن تمام سفارشات یک کابر در ماه جاری
\item محاسبه میزان فروش یک شعبه در یک ماه
\item پیدا کردن کاربرهایی با بیشترین مقدار ثبت سفارش در یک ماه

\end{enumerate}


\end{document}
}